\documentclass[11pt]{report}
\title{MATH-36032 PSBC Project 1: Three short questions}
\author{Suvineet Singh Kalsi \\ ID: 9844719}
\date{}

\setlength\parindent{0pt}

\usepackage[utf8]{inputenc}
\usepackage{amsmath}
\usepackage{amssymb}
\usepackage{amsthm}
\usepackage{listings}
\usepackage[margin=0.8in]{geometry}
\usepackage{xcolor}

 
\definecolor{codegreen}{rgb}{0,0.6,0}
\definecolor{codegray}{rgb}{0.5,0.5,0.5}
\definecolor{codepurple}{rgb}{0.58,0,0.82}
\definecolor{backcolour}{rgb}{0.95,0.95,0.92}
 
\lstdefinestyle{mystyle}{
    backgroundcolor=\color{backcolour},
    commentstyle=\color{codegreen},
    keywordstyle=\color{blue},
    numberstyle=\tiny\color{codegray},
    stringstyle=\color{codepurple},
    basicstyle=\ttfamily\footnotesize,
    breakatwhitespace=false,
    breaklines=true,
    captionpos=b,
    keepspaces=true,
    numbers=left,
    numbersep=5pt,
    showspaces=false,
    showstringspaces=false,
    showtabs=false,
    tabsize=2,
}
\renewcommand{\chaptername}{Question}
\lstset{style=mystyle}

\begin{document}
\maketitle

% ******************************** Question 1 ******************************** %
\chapter{Cubic Taxicab number}

\section*{Problem}
\paragraph{Cubic Taxicab number:}
is a positive integer which can be written in two or more distinct ways of the form:
\begin{equation*}
	t = a^3 + b^3
\end{equation*} where $a$, $b$ $\in$ $\mathbb{Z^+}$ \\

Write a function \texttt{CubicTaxicabNum(N)} that takes an input $N$ and returns the smallest cubic taxicab number which is greater than or equal to $N$.

\section{Approach}
Suppose we have a a function \texttt{isCubicTaxiCab(X)} which returns a \texttt{boolean} value determining whether $X \in \mathbb{Z^+}$ is a taxicab number. We will assume the input $N$ is always a positive integer. Using this function we can find the smallest cubic taxicab number greater than or equal to $N$ by checking each integer greater than or equal to $N$, until we find a cubic taxicab number. \\

\lstinputlisting[language=Matlab]{CubicTaxicabNum.m}

Now we can implement the function \texttt{isCubicTaxiCab(X)}. First, we make an observation about the solution. Assume $t \in \mathbb{Z^+}$ is a cubic taxicab number. Then:

\paragraph{Observation:}
If $t=a^3+b^3$ for $a, b \in \mathbb{Z^+}$ then assuming without loss of generality that $a \leqslant b$ we have:
\begin{equation*}
	a \leqslant b \leqslant \texttt{floor(}\sqrt[3]{t}\texttt{)} 
\end{equation*} where \texttt{floor($\sqrt[3]{t}$)} is the truncated value of the cube root of $t$. \\ 
e.g. $t=20 \implies$ \texttt{floor($\sqrt[3]{20}$)} $=$ \texttt{floor(2.714...))} $=2$. \\

It follows from the observation that it is sufficient to check numbers in the range $[1,$ \texttt{floor($\sqrt[3]{t}$)}$]$ as possible candidates for $a$ and $b$ such that $t=a^3+b^3$. We claim the following is a solution for the function \texttt{isCubicTaxiCab(X)}: \\

\lstinputlisting[language=Matlab]{isCubicTaxiCab.m}

\section{Analysis}
\subsection{Correctedness:}
\begin{proof}[\unskip\nopunct]
Let $Y=$ \texttt{floor($\sqrt[3]{X}$)}. The above function attempts to find two distinct combinations from the range $A=[1,Y]\subset \mathbb{Z^+}$ for which the sum of cubes is equal to $X$. \\ 

The algorithm first checks if $1^3+Y^3=X$ i.e ($A(1)^3+A(Y)^3 = X$). If this is the case then we have found one combination whose sum of cubes is equal to $X$ (line 15-18). However, if the sum of cubes is greater than $X$ then we must add a smaller value to $1^3$ to get closer to $X$. Hence, we then decrement $j$ by 1 (line 13). With a similar argument, if the sum of cubes is less than $X$ then we must increment $i$ by one (line 14). \\

This is repeated until either $i=j$ or we have found 2 combinations (condition in the \texttt{while} loop). The latter implies that $X$ is a cubic taxicab number, whereas if $i=j$ before $\texttt{comb\_count}=2$ then we can conclude that $X$ is not a cubic taxicab number (line 22-26). \\

The algorithm checks all possible combinations in the list $A = [1,Y] \subset \mathbb{Z^+}$. Therefore, we will always find two combinations if $X$ is a cubic taxicab number and will only return \texttt{false} when it isn't. Hence, the algorithm is correct.
\end{proof}

\subsection{Efficiency:}
\subsubsection{\texttt{CubicTaxicabNum(N)}:}
This function clearly iterates $n$ times where $n$ is the difference between $N$ and the smallest cubic taxicab number greater than or equal to $N$.

\subsubsection{\texttt{isCubicTaxiCab(N)}:}
This function loops at most $Y=$ \texttt{floor($\sqrt[3]{X}$)} times, in the case when $X$ is \textbf{not} a cubic taxicab number. The algorithm checks all possible combinations in the list $A=[1,Y]\subset\mathbb{Z^+}$ and terminates without finding at least two correct combinations. \\

This implementation is much efficient than a ``brute force'' approach where we would check every possible combination in $A$ which can happen at most $Y^2$. \\

We can say our solution runs in \textit{``linear time complexity in $Y$''} which is significantly faster than the "brute force" approach which is \textit{``polynomial time complexity in $Y$''}, especially for very large input $X$.

\section{Results}
Below are the following results for two inputs $N=1$ and $N=36032$. \\

\begin{lstlisting}[title={N=1}]
>> CubicTaxicabNum(1)
     1    12
     9    10
ans = 1729
\end{lstlisting}

This is a correct result because 1729 is the first cubic taxicab number, associated with an anecdote about Ramanujan by G. H. Hardy.

\begin{equation*}
	1729 = 1^3 + 12^3 = 9^3 + 10^3
\end{equation*}

\textbf{NB:} The above result is run by uncommenting line 22 in the \texttt{isCubicTaxiCab(X)} function above. This also outputs the first two combinations whose cube sum is equal to the cubic taxicab number found. \\
\begin{lstlisting}[title={N=36032}]
>> CubicTaxicabNum(36032)
     2    34
    15    33
ans = 39312
\end{lstlisting}
Therefore, $\textbf{39312} = 2^3 +34^3 = 15^3 + 33^3 = ...$ (possibly more) is the smallest Cubic Taxicab number greater than or equal to 36032


% ******************************** Question 2 ******************************** %
\chapter{Catalan's Constant}
\section*{Problem}
Here we have the Catalan's constant:

\begin{equation*}
	G = \sum_{k=0}^{\infty} \frac{(-1)^k}{(2k +1)^2} = \frac{1}{1^2} - \frac{1}{3^2} + \frac{1}{5^2} + ... + \approx 0.915965594177219
\end{equation*}

$G$ can expressed in terms of various sums and series or special integrals, however, hasn't yet been proved to be irrational or not. \\

Write a function \texttt{RatAppCat(N)} which takes an input $N\in\mathbb{Z^+}$ and returns a pair $(p, q)$ such that $p,q\in\mathbb{Z^+}$ and $p/q$ is the best rational approximation of $G$. i.e. $|p/q-G|$ is the smallest. We add a constraint that $p+q\leqslant N$.


\section{Approach}

\paragraph{Observation:}
For a given $q\in\mathbb{Z^+}$, we can find the \textit{accurate} $p^*\in\mathbb{R^+}$ such that $p^*=qG$. Then for any such $p^*$, $p=\texttt{round(}p^*\texttt{)} \in \mathbb{Z^+}$ will give the best approximation of $p/q =G$ for the given $q$. \\

Moreover, it is trivial to see that  $p_0 + q_0 > N \implies p_1 + q_1 > N$, where $p_0, p_1, q_0, q_1 \in \mathbb{Z^+}$ s.t $p_0/q_0$ and $p_1/q_1$ are approximations of $G$ with $p_0 \leq p_1$ and $q_0 \leq q_1$. \\

Using the first observation it is sufficient to iterate over all the values of $q\in[1,N]$ such that $p+q\leq N$ where $p = \texttt{round(}qG\texttt{)} \in \mathbb{Z^+}$. On each iteration, we can calculate $d = |p/q - G|$, whilst keeping track of the smallest $d$. Then using the second observation, we can stop iterating over $q$ any further when we get the first $p+q>N$. \\

The following function \texttt{RatAppCat(N)} implements the above approach:
\lstinputlisting[language=Matlab]{RatAppCat.m}

\section{Analysis}
\subsection{Correctedness:}
\begin{proof}[\unskip\nopunct]
The algorithm loops from \texttt{qi = 1 to N}. For each \texttt{qi} the \textit{rounded perfect value} \texttt{rvp = round(G*qi)} is calculated. Which we know will be the best value of $p$ for the given \texttt{qi} from our observation. The algorithm then checks if the current \texttt{rvp + q > N}. In the case where this is true, the algorithm halts because we can no longer find a better solution whilst obeying our restriction that $p+q\leq N$. \\

However, if this condition is false, then $d=|p/q - G|$ is calculated (for \texttt{qi}) and if this is smaller than any previous $d$ we pick this combination to be the best $p$ and $q$. \\
\end{proof}

\subsection{Efficiency:}
The algorithm approximately loops at most $N/2$ times because for a given $q$, $p = \texttt{round(}qG\texttt{)}$ will be ``very close'' to $q$ as $G\approx 0.91596559417721$. Therefore, if \texttt{qi $\approx N/2$} then $\texttt{qi + rvp} \approx \texttt{qi + qi} \approx 2\times N/2 = N$ then $\texttt{qi+1 + rvp} \gtrapprox N$. \\

Therefore, the algorithm runs in \textit{``linear time complexity in $N$''}, however, it does approximately $N/2$ operations rather than $N$. 

\section{Results} 
Below is the result for $N=2018$. \\

\begin{lstlisting}[title={N=2018}]
>> [p,q] = RatAppCat(2018)
p = 109
q = 119

>> p/q 
ans = 0.915966386554622
\end{lstlisting}
\end{document}
