\documentclass[12pt]{report}
\title{MATH-36032 PSBC Project 1: Three short questions}
\author{Suvineet Singh Kalsi \\ ID: 9844719}
\date{}

\setlength\parindent{0pt}

\usepackage[utf8]{inputenc}
\usepackage{amsmath}
\usepackage{amssymb}
\usepackage{amsthm}
\usepackage{listings}
\usepackage[utf8]{inputenc}
\usepackage[margin=1.0in]{geometry}
\usepackage{xcolor}

 
\definecolor{codegreen}{rgb}{0,0.6,0}
\definecolor{codegray}{rgb}{0.5,0.5,0.5}
\definecolor{codepurple}{rgb}{0.58,0,0.82}
\definecolor{backcolour}{rgb}{0.95,0.95,0.92}
 
\lstdefinestyle{mystyle}{
    backgroundcolor=\color{backcolour},
    commentstyle=\color{codegreen},
    keywordstyle=\color{blue},
    numberstyle=\tiny\color{codegray},
    stringstyle=\color{codepurple},
    basicstyle=\ttfamily\footnotesize,
    breakatwhitespace=false,
    breaklines=true,
    captionpos=b,
    keepspaces=true,
    numbers=left,
    numbersep=5pt,
    showspaces=false,
    showstringspaces=false,
    showtabs=false,
    tabsize=2,
}
\renewcommand{\chaptername}{Question}
\lstset{style=mystyle}

\begin{document}
\maketitle

% ----- Chapter 1 - cubic taxicab number ----- %
\chapter{Cubic Taxicab number}

\section*{Problem}
\paragraph{Cubic Taxicab number:}
is a positive integer which can be written in two or more distinct ways of the form:
\begin{equation*}
	t = a^3 + b^3
\end{equation*} where $a$, $b$ $\in$ $\mathbb{Z^+}$ \\

Write a function \texttt{CubicTaxicabNum(N)} that takes an input $N$ and returns the smallest cubic taxicab number which is greater than or equal to $N$.

\section{Approach}
Suppose we have a a function \texttt{isCubicTaxiCab(X)} which returns a \texttt{boolean} value determining whether $X \in \mathbb{Z^+}$ is a taxicab number. We will assume the input $N$ is always a positive integer. Using this function we can find the smallest cubic taxicab number greater than or equal to $N$ by checking each integer greater than or equal to $N$, until we find a cubic taxicab number. \\

\begin{lstlisting}[language=Matlab]
function ctn = CubicTaxicabNum(N)
% CUBICTAXICABNUM return the smallest cubic taxicab number greater than
% or equal to N

ctn = N;
while (~isCubicTaxiCab(ctn))
    ctn = ctn + 1;
end
end
\end{lstlisting}

Now we can implement the function \texttt{isCubicTaxiCab(X)}. First, we make an observation about the solution. Assume $t \in \mathbb{Z^+}$ is a cubic taxicab number. Then:

\paragraph{Observation:}
If $t=a^2 +b^2$ for $a, b \in \mathbb{Z^+}$ then assuming without loss of generality that $a \leqslant b$ we have:
\begin{equation*}
	a \leqslant b \leqslant \texttt{floor(}\sqrt[3]{t}\texttt{)} 
\end{equation*} where \texttt{floor($\sqrt[3]{t}$)} is the truncated value of the cube root of $t$. \\ 
e.g. $t=20 \implies$ \texttt{floor($\sqrt[3]{20}$)} $=$ \texttt{floor(2.714...))} $=2$. \\

It follows from the observation that it is sufficient to check numbers in the range $[1,$ \texttt{floor($\sqrt[3]{t}$)}$]$ as possible candidates for $a$ and $b$ such that $t=a^3+b^3$. We claim the following is a solution for the function \texttt{isCubicTaxiCab(X)}: \\

\begin{lstlisting}[language=Matlab]
function x = isCubicTaxiCab(X)
% ISCUBICTAXICAB returns a boolean value determining if X is a cubic
% taxicab number or not.

i = 1; j = floor(nthroot(X, 3));
comb_count = 0;
A = 1:j;
x = false;
combo = zeros(2); % Tracks the first two combinations if x is ctn
while (i < j)
    cube_sum = A(i)^3 + A(j)^3;
    if (cube_sum > X)
        j = j - 1;
    elseif (cube_sum < X)
        i = i + 1;
    else
        comb_count = comb_count + 1;
        combo(comb_count,:) = [i j];
        i = i + 1; j = j - 1;
    end
end
if (comb_count == 2)
    x = true;
    % disp(combo); % uncomment to see the first 2 sum of cubes.
    return;
end
end
\end{lstlisting}

\section{Analysis}
\subsection{Correctedness:}
\begin{proof}[\unskip\nopunct]
Let $Y=$ \texttt{floor($\sqrt[3]{X}$)}. The above function attempts to find two distinct combinations from the range $A=[1,Y]\subset \mathbb{Z^+}$ for which the sum of cubes is equal to $X$. \\ 

The algorithm first checks if $1^3+Y^3=X$ i.e ($A(1)^3+A(Y)^3 = X$). If this is the case then we have found one combination whose sum of cubes is equal to $X$ (line 15-17). However, if the sum of cubes is greater than $X$ then we must add a smaller value to $1^3$ to get closer to $X$. Hence, we then decrement $j$ by 1 (line 12). With a similar argument, if the sum of cubes is less than $X$ then we must increment $i$ by one (line 14). \\

This is repeated till either $i=j$ (condition in the \texttt{while} loop) or we have found 2 combinations (line 20-22). The latter implies that $X$ is a cubic taxicab number, whereas if $i=j$ before $\texttt{comb\_count}=2$ then we can conclude that $X$ is not a cubic taxicab number. \\

The algorithm checks all possible combinations in the list $A = [1,Y] \subset \mathbb{Z^+}$. Therefore, we will always find two combinations if $X$ is a cubic taxicab number and will only return \texttt{false} when it isn't. Hence, the algorithm is correct.
\end{proof}

\subsection{Efficiency:}
\subsubsection{\texttt{CubicTaxicabNum(N)}:}
This function clearly iterates $n$ times where $n$ is the difference between $N$ and the smallest cubic taxicab number greater than or equal to $N$.

\subsubsection{\texttt{isCubicTaxiCab(N)}:}
This function loops at most $Y=$ \texttt{floor($\sqrt[3]{X}$)} times, in the case when $X$ is \textbf{not} a cubic taxicab number. The algorithm checks all possible combinations in the list $A=[1,Y]\subset\mathbb{Z^+}$ and terminates without finding at least two correct combinations. \\

This implementation is much efficient than a "brute force" approach where we would check every possible combination in $A$ which can happen at most $Y^2$. \\

We can say our solution runs in \textit{"linear time complexity in $Y$"} which is significantly faster than the "brute force" approach which is \textit{"polynomial time complexity in $Y$"}, especially for very large input $X$.

\section{Result}
Below are the following results for two inputs $N=1$ and $N=36032$. \\

\begin{lstlisting}[title={N=1}]
input:
CubicTaxicabNum(1)

output:
>> CubicTaxicabNum(1)
     1    12
     9    10

ans = 1729
\end{lstlisting}

This is a correct result because 1729 is the first cubic taxicab number, associated with an anecdote about Ramanujan by G. H. Hardy.

\begin{equation*}
	1729 = 1^3 + 12^3 = 9^3 + 10^3
\end{equation*}

\textbf{NB:} The above result is run by uncommenting line 24 in the \texttt{isCubicTaxiCab(X)} function above. This also outputs the first two combinations whose cube sum is equal to cubic taxicab number found. \\
\begin{lstlisting}[title={N=36032}]
input:
CubicTaxicabNum(36032)

output:
>> CubicTaxicabNum(36032)
     2    34
    15    33

ans = 39312
\end{lstlisting}
Therefore, $\textbf{39312} = 2^3 +34^3 = 15^3 + 33^3 = ...$ (possibly more) is the smallest Cubic Taxicab number greater than or equal to 36032
\end{document}
